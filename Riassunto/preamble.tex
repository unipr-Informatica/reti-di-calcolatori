\usepackage{iftex}              % Consente di verificare l'engine in uso
\ifPDFTeX
    \usepackage[utf8]{inputenc}     % Permette di inserire qualunque carattere UTF-8
\fi
\usepackage[table,xcdraw,dvipsnames]{xcolor}
\usepackage{tikz}               % Utilizzata per disegnare schemi
\usepackage{amsmath}            % Utilizzata per inserire formule matematiche
\usepackage{hyperref}           % Serve a generare la tabella dei contenuti (TOC)
\usepackage{enumitem}           % Serve a definire lo stile dell'itemize
\usepackage[italian]{babel}     % Serve per la corretta hyphenation delle parole
\usepackage{caption}            % Serve a rimuovere la caption di default delle tabelle
\usepackage{enumitem}           % Serve a personalizzare le enumerate
\usepackage{listings}           % Serve per i code snippets
\usepackage{verbatim}           % Serve per gli \verb
\usepackage{graphicx}           % Serve a specificare width e height a un'immagine storta del capitolo quattro

\usetikzlibrary{calc,decorations.pathreplacing}